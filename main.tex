\documentclass{article}
\usepackage[utf8]{inputenc}
\usepackage{multirow}
\newlength\drop
\makeatletter
\newcommand*\titleM{\begingroup% Misericords, T&H p 153
\setlength\drop{0.08\textheight}
\centering
\vspace*{\drop}
{\Huge\bfseries Niklas Karvonen}\\[\baselineskip]
{\scshape Resumé}\\[\baselineskip]
\vfill
{\large\scshape Application for Postdoc @ EISLAB (Ref 3890-2018)}\par
\vfill
{\scshape \@date}\par
\vspace*{2\drop}
\endgroup}
\makeatother

\begin{document}
% Title page
\begin{titlingpage}
\titleM
\end{titlingpage}
\newpage

% TOC
\tableofcontents
\newpage

% Personal Information
\section{Personal Information}


\subsection{Full name} Niklas Karvonen  \\
\subsection{Date of birth} 1979-10-21 (8911) \\
\subsection{Address, Phone number, email, webpage}
Karbinvägen 2A, 975 93 Luleå \\
070 - 55 444 16 \\ 
niklas.karvonen@gmail.com \\
http://www.niklaskarvonen.com \\

\subsection{Current employment} Researcher @ Eislab, Luleå University of Technology \\
\subsection{Previous employments}
2013-2018 Doctoral Student @ Luleå University of Technology \\
2011-2013 Head Developer @ Taxijakt AB \\
2010-2011 CEO @ Trezer AB \\
2008-2010 CTO @ KYAB AB \\

\subsection{Other information}

\newpage

% Degrees
\section{Degrees}
\subsection{Master of Science Degree}
Luleå University of Technology 2010 - Thesis title: "Time-Efficient Algorithms for Laser Guided Autonomous Driving" \\
\subsection{Licentiate Degree}
Luleå University of Technology 2015 - Thesis title: "Activity Recognition in Resource-Constrained Pervasive Systems" \\

\subsection{Doctoral Degree}
Luleå University of Technology 2018 - Thesis title: "Unobtrusive Activity Recognition in Resource-Constrained Environments" \\

\subsection{Associate Professor}

\subsection{Other Degrees}
\newpage

% Research merits
\section{Research Merits}
\subsection{Research Profile}
My research has mainly involved using resource-constrained computer systems to build small, power-efficient, and smart wearable systems. In my work I have worked with computer programming ranging from embedded systems to web. The methodology has been practical in nature, with the aim to have results that can be readily implemented in real-life applications.

\subsection{Research Projects}
No project work has been performed after PhD graduation. \\(graduation date: 2018-12-11)

\subsection{Planned Research}
Continued research on Internet of Things and resource-constrained devices.

\subsection{Publications}
\subsubsection{Peer-reviewed Articles in Scientific Journals}
\textbf{ "Classifier Optimized for Resource-constrained Pervasive Systems and Energy-efficiency"}, 
Niklas Karvonen, Lara Lorna Jimenez, Miguel Simon Gomez, Joakim Nilsson, Basel Kikhia, Josef Hallberg
In \textit{International Journal of Computational Intelligence Systems}, ISSN 1875-6891, E-ISSN 1875-6883, Vol. 10, no 1, p. 1272-1279, 
Published by Atlantis Press, 
DOI 10.2991/ijcis.10.1.86, 
2017.\\
\textbf{Contribution:} Design of the study, implementation of the algorithm, analysis of results, writing of article.
\\ \\

\textbf{ "Utilizing a Wristband Sensor to Measure the Stress Level for People with Dementia"},
Basel Kikhia, Thanos G. Stavropoulos, Stelios Andreadis, Niklas Karvonen, Ioannis Kompatsiaris, Stefan Sävenstedt, Marten Pijl, Catharina Melander
In \textit{Sensors}, ISSN 1424-8220, E-ISSN 1424-8220, Vol. 16, no 12, Article number 1989,
Published by MDPI, ISSN 1424-8220,
DOI https://doi.org/10.3390/s16121989,
24 November 2016.\\
\textbf{Contribution:} Data processing and contributions to the writing of the article.

\textbf{"Analyzing Body Movements within the Laban Effort Framework Using a Single Accelerometer."}, 
Basel Kikhia, Miguel Gomez, Lara Lorna Jimenez, Josef Hallberg, Niklas Karvonen, and Kåre Synnes.
In \textit{ Sensors}, Volume 14, Issue 3, Pages 5725-5741, 
Published by MDPI, ISSN 1424-8220, 
DOI 10.3390/s140305725, 
21 March 2014. \\
\textbf{Contribution:} Related work, writing of the article, minor role in analysis and discussion.
\\ \\

\subsubsection{Books}

\subsubsection{Peer-reviewed Articles in Conferences}
\textbf{ "A Domain Knowledge-Based Solution for Human Activity Recognition: The UJA Dataset Analysis"},
Niklas Karvonen, Denis Kleyko
In \textit{ Multidisciplinary Digital Publishing Institute Proceedings}, Vol. 2, Issue 19, Article number 1261, 
Published by MDPI, EISSN 2504-3900, 
DOI 10.3390/proceedings2191261, 
19 October 2018.\\
\textbf{Contribution:} Design of the study, analysis of results, writing of article.
\\ \\


\textbf{Computationally Inexpensive Classifier Merging Cellular Automata and MCP-Neurons.},
Karvonen N., Kikhia B., Jiménez L.L., Gómez Simón M., Hallberg J.
In: \textit{García C., Caballero-Gil P., Burmester M., Quesada-Arencibia A. (eds) Ubiquitous Computing and Ambient Intelligence. IWAAL 2016, AmIHEALTH 2016, UCAmI 2016. Lecture Notes in Computer Science, vol 10070. Springer, Cham.}
DOI 10.1007/978-3-319-48799-1\_42.\\
\textbf{Contribution:} Design of the study, implementation of the algorithm, analysis of results, writing of article.
\\ \\

\textbf{ "Designing ICT for Health and Wellbeing - An Allostatic, Behavioral-Change Approach to a Monitoring and Coaching App"}, 
Anders Hedman, Niklas Karvonen, Josef Hallberg, Juho Merilahti
In \textit{Proceedings of the 6th International Workshop on Ambient Assisted Living (IWAAL)}, Lecture Notes in Computer Science (Information Systems and Applications, incl. Internet/Web, and HCI), Ambient Assisted Living and Daily Activities, Volume 8868, Pages 244-251, 
Published by Springer, ISSN 0302-9743, Belfast, UK, 
2-5 December 2014.\\
\textbf{Contribution:} Design and implementation of user-interfaces. Design of the implementation of the behavioural-change model. Writing the article.
\\ \\

\subsubsection{Other Publications}
\newpage

\subsection{Research Grants}
720.000 SEK funding from ALMI and Region Norrbotten for developing "Sensorizon", a commercialisation of the research results from the paper "Utilizing a Wristband Sensor to Measure the Stress Level for People with Dementia". I participated in the writing of the application and worked as the head of development within the project.

\subsection{Research Collaborations}

\subsection{Dissemination of Research Results}

\subsection{Awards}
Venture Cup regional final winners for a business idea using wearable sensors to detect stress in persons with dementia. This was a commercialisation of our previous study in the paper "Utilizing a Wristband Sensor to Measure the Stress Level for People with Dementia".

\subsection{Other Research Merits}
\newpage


% Pedagogical merits
\section{Pedagogics}
\subsection{Pedagogical Education}
University Pedagogics I - LTU (3.5 HP).\\
University Pedagogics II - LTU (4 HP - Ongoing).

\subsection{Pedagogical Reflection}
When I started working as a teacher, I used very few pedagogical tools other than methods I had used when teaching myself. I was unaware of many of the principles at play in the learning process, including the study environment, role models, the relationship between the teaching activity and the class, the motivation of students, and the work students perform outside of the classroom.

During my teaching, it has become evident that my preconception of the teacher’s role has been focused on a one-way communication from the teacher to the student. The “traditional lecture” as a teaching and learning activity (TLA) therefore earlier dominated my view on teaching. Naturally, my own education have biased me in this matter since traditional lectures played a central role then. I still believe traditional lectures are good TLAs, but I only using them when I see them fit the context. A central part in my teaching now instead involves different forms of \textit{interaction} with the students, preferably in smaller groups (e.g. labs and discussions). I also strive to incorporate practical work whenever possible in order for theory to be put into use.

My philosophy on teaching is that the knowledge and wisdom from another human being, shared in a personal way, has the power to change another person. Becoming a teacher should therefore be honoured together with the trust and responsibility that comes with that role. I believe that the most rewarding teaching and best gained knowledge, is formed with mutual and high expectations from both the students and the teacher.

\subsection{Teaching Experience}
Lab assistant - Object Oriented Programming. (LTU) \\
Lab assistant - Real-Time Systems. (LTU) \\
Lecturer - Dynamic Web Systems. (LTU) \\

\subsection{Supervision Experience}
Master Thesis Supervisor - Tobias Axelsson (LTU). \\
Master Thesis Supervisor - Simon Nilsson Guldstrand, Maxime Koitsalu (LTU). \\

\subsection{Course Materials}

\subsection{Course Planning}

\subsection{Pedagogical Collaborations}

\subsection{Pedagogical Awards}


\subsection{Other Pedagogic Experience}
Lecturer - ASP.NET programming (Strand Interconnect, CopyCat, Tanzania)
Supervisor - FAS 3 Kurt Kopppari (Trezer AB).\\


\section{Management Experience}
Chief Executive Officer - Memorizon AB (2017-2018).\\
Chief Technical Officer - KYAB (2009-2010).

\section{Other Assignments}
\subsection{Board Representation}
Member of the board in Memorizon AB. \\
PhD Representative for SRT (LTU) in the PhD Association.

\subsection{Ownership in Private Companies}
Memorizon AB (5\% shares, passive ownership).

\subsection{Non-profit Work}

\section{References}
Josef Hallberg (LTU). \\
Kåre Synnes (LTU). \\
Peter Parnes (LTU). \\
Fredrik Bengtsson (LTU). \\

\section{Publications Relevant to the Position}
See Section 3.4.

\end{document}
