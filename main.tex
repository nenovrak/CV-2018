\documentclass{article}
\usepackage[utf8]{inputenc}
\usepackage{multirow}
\newlength\drop
\makeatletter
\newcommand*\titleM{\begingroup% Misericords, T&H p 153
\setlength\drop{0.08\textheight}
\centering
\vspace*{\drop}
{\Huge\bfseries The Title}\\[\baselineskip]
{\scshape the subtitle}\\[\baselineskip]
\vfill
{\large\scshape the author}\par
\vfill
{\scshape \@date}\par
\vspace*{2\drop}
\endgroup}
\makeatother

\begin{document}
% Title page
\begin{titlingpage}
\titleM
\end{titlingpage}

% TOC
\tableofcontents
\newpage

% Personal Information
\section{Personal Information}

\begin{table}[ht]
\begin{tabular}[t]{lllll}
Full name & Niklas Karvonen  \\
Date of birth & 1979-10-21 (8911) \\
Phone number & 070-55 444 16 \\ 
E-mail & niklas.karvonen@gmail.com \\
Website & http://www.niklaskarvonen.com \\
Postal Address & Karbinvägen 2A, 975 93 Luleå \\
Current employment & Researcher @ Eislab, Luleå University of Technology \\
\multirow[t]{4}{*}{Previous employments}
& 2013-2018 Doctoral Student @ Luleå University of Technology \\
& 2011-2013 Head Developer @ Taxijakt AB \\
& 2010-2011 CEO @ Trezer AB \\
& 2008-2010 CTO @ KYAB AB \\
\end{tabular}
\end{table}
\newpage

% Degrees
\section{Degrees}
\begin{table}[ht]
\begin{tabular}[t]{p{3cm}p{5cm}p{5cm}}
\textbf{Degree} & \textbf{Issuer} & \textbf{Additional Information}\\
PhD & Luleå University of Technology & Thesis title: "Unobtrusive Activity Recognition in Resource-Constrained Environments" \\
Licentiate & Luleå University of Technology & Thesis title: "Activity Recognition in Resource-Constrained Pervasive Systems" \\
Master of Science & Luleå University of Technology & Thesis title: "Time-Efficient Algorithms for Laser Guided Autonomous Driving" \\
\end{tabular}
\end{table}

% Research merits
\section{Research Merits}
\subsection{Research Profile}
My research has mainly involved using resource-constrained computer systems to build small, power-efficient, and smart wearable systems. In my work I have worked with computer programming ranging from embedded systems to web. The methodology has been practical in nature, and the aim is to have results be readily implemented in real-life applications.

\subsection{Research Projects}
No project work has been performed after PhD graduation.

\subsubsection{Planned Research}
Continued research on how to make more efficient systems using resource-constrained devices.

\subsubsection{Publications}
\textbf{"Analyzing Body Movements within the Laban Effort Framework Using a Single Accelerometer."}, 
Basel Kikhia, Miguel Gomez, Lara Lorna Jimenez, Josef Hallberg, Niklas Karvonen, and Kåre Synnes.
In \textit{ Sensors}, Volume 14, Issue 3, Pages 5725-5741, 
Published by MDPI, ISSN 1424-8220, 
DOI 10.3390/s140305725, 
21 March 2014. \\
\textbf{Contribution:} Related work, writing of the article, minor role in analysis and discussion.
\\ \\

\textbf{ "A Domain Knowledge-Based Solution for Human Activity Recognition: The UJA Dataset Analysis"},
Niklas Karvonen, Denis Kleyko
In \textit{ Multidisciplinary Digital Publishing Institute Proceedings}, Vol. 2, Issue 19, Article number 1261, 
Published by MDPI, EISSN 2504-3900, 
DOI 10.3390/proceedings2191261, 
19 October 2018.\\
\textbf{Contribution:} Design of the study, analysis of results, writing of article.
\\ \\

\textbf{ "Classifier Optimized for Resource-constrained Pervasive Systems and Energy-efficiency"}, 
Niklas Karvonen, Lara Lorna Jimenez, Miguel Simon Gomez, Joakim Nilsson, Basel Kikhia, Josef Hallberg
In \textit{International Journal of Computational Intelligence Systems}, ISSN 1875-6891, E-ISSN 1875-6883, Vol. 10, no 1, p. 1272-1279, 
Published by Atlantis Press, 
DOI 10.2991/ijcis.10.1.86, 
2017.\\
\textbf{Contribution:} Design of the study, implementation of the algorithm, analysis of results, writing of article.
\\ \\

\textbf{ "Low-Power Classification using FPGA - An Approach based on Cellular Automata and Hyperdimensional Computing"}, 
Niklas Karvonen, Joakim Nilsson, Denis Kleyko, Lara Lorna Jimenez,
Submitted to \textit{ IEEE Pervasive Computing}, ISSN 1536-1268, 
November 2018.\\
\textbf{Contribution:} Design of the study, implementation of the algorithm, analysis of results, writing of article.
\\ \\

\textbf{Computationally Inexpensive Classifier Merging Cellular Automata and MCP-Neurons.},
Karvonen N., Kikhia B., Jiménez L.L., Gómez Simón M., Hallberg J.
In: \textit{García C., Caballero-Gil P., Burmester M., Quesada-Arencibia A. (eds) Ubiquitous Computing and Ambient Intelligence. IWAAL 2016, AmIHEALTH 2016, UCAmI 2016. Lecture Notes in Computer Science, vol 10070. Springer, Cham.}
DOI 10.1007/978-3-319-48799-1\_42.\\
\textbf{Contribution:} Design of the study, implementation of the algorithm, analysis of results, writing of article.
\\ \\

\textbf{ "Designing ICT for Health and Wellbeing - An Allostatic, Behavioral-Change Approach to a Monitoring and Coaching App"}, 
Anders Hedman, Niklas Karvonen, Josef Hallberg, Juho Merilahti
In \textit{Proceedings of the 6th International Workshop on Ambient Assisted Living (IWAAL)}, Lecture Notes in Computer Science (Information Systems and Applications, incl. Internet/Web, and HCI), Ambient Assisted Living and Daily Activities, Volume 8868, Pages 244-251, 
Published by Springer, ISSN 0302-9743, Belfast, UK, 
2-5 December 2014.\\
\textbf{Contribution:} Design and implementation of user-interfaces. Design of the implementation of the behavioural-change model. Writing the article.
\\ \\

\textbf{ "Utilizing a Wristband Sensor to Measure the Stress Level for People with Dementia"},
Basel Kikhia, Thanos G. Stavropoulos, Stelios Andreadis, Niklas Karvonen, Ioannis Kompatsiaris, Stefan Sävenstedt, Marten Pijl, Catharina Melander
In \textit{Sensors}, ISSN 1424-8220, E-ISSN 1424-8220, Vol. 16, no 12, Article number 1989,
Published by MDPI, ISSN 1424-8220,
DOI https://doi.org/10.3390/s16121989,
24 November 2016.\\
\textbf{Contribution:} Data processing and contributions to the writing of the article.

\subsubsection{Research Grants}
Approx. 750.000 SEK funding from ALMI and Region Norrbotten for developing "Sensorizon", a commercialisation of the research results from the paper "Utilizing a Wristband Sensor to Measure the Stress Level for People with Dementia".


\end{document}
