\documentclass{article}
\usepackage[utf8]{inputenc}
\usepackage{multirow}
\newlength\drop
\makeatletter
\newcommand*\titleM{\begingroup% Misericords, T&H p 153
\setlength\drop{0.08\textheight}
\centering
\vspace*{\drop}
{\Huge\bfseries Niklas Karvonen}\\[\baselineskip]
{\scshape Resumé}\\[\baselineskip]
\vfill
{\large\scshape }\par
\vfill
{\scshape \@date}\par
\vspace*{2\drop}
\endgroup}
\makeatother

\begin{document}
% Title page
\begin{titlingpage}
\titleM
\end{titlingpage}
\newpage

% TOC
\tableofcontents
\newpage

\section{Summary}


% Personal Information
\section{Personal Information}


\subsection{Full name} Niklas Karvonen  \\
\subsection{Date of birth} 1979-10-21 (8911) \\
\subsection{Address, Phone number, email, webpage}
Karbinvägen 2A, 975 93 Luleå \\
070 - 55 444 16 \\ 
niklas.karvonen@gmail.com \\

\subsection{Current employment} Postdoc Researcher @ Eislab, Luleå University of Technology \\
\subsection{Previous employments}
2013-2018 Doctoral Student @ Luleå University of Technology \\
2011-2013 Self-employed Software Consultant @ Stay Eleven AB \\
2010-2011 CEO @ Trezer AB \\
2008-2010 CTO @ KYAB AB \\

\subsection{Other information}

\newpage

% Degrees
\section{Degrees, assessments and evaluations}
\subsection{Doctoral Degree}
Luleå University of Technology 2018 - Thesis title: "Unobtrusive Activity Recognition in Resource-Constrained Environments" \\

\subsection{Licentiate Degree}
Luleå University of Technology 2015 - Thesis title: "Activity Recognition in Resource-Constrained Pervasive Systems" \\

\subsection{Master of Science Degree}
Luleå University of Technology 2010 - Thesis title: "Time-Efficient Algorithms for Laser Guided Autonomous Driving" \\

\subsection{Other}
\begin{description}
\item Semi-professional drummer 2010 (ongoing) - "Machinae Supremacy" band
\item Venture Cup regional final winners in 2015 (Memorizon AB)
\item Selected for the Young Entrepreneurs in Barents project 2008.
\item Selected as a promising young entrepreneur in 2005 by Innovationsbron
\end{description}


\newpage

% Research merits
\section{Research Merits}
\subsection{Research Profile}
My research has mainly involved using resource-constrained computer systems to build small, power-efficient, and smart wearable systems. In my work I have worked with computer programming ranging from embedded systems to web. The methodology has been practical in nature, with the aim to have results that can be readily implemented in real-life applications.

%\subsection{Research Projects}
%No project work has been performed after PhD graduation. \\(graduation date: 2018-12-11)

%\subsection{Planned Research}
%Continued research on Internet of Things and resource-constrained devices.

\subsection{Publications}
\subsubsection{Peer-reviewed Scientific Articles}

\begin{description}
\item[Paper 1]
Basel Kikhia, Miguel Gomez, Lara Lorna Jimenez, Josef Hallberg, Niklas Karvonen, and K\aa re Synnes.\\
{\bf "Analyzing Body Movements within the Laban Effort Framework Using a Single Accelerometer."}, 
In {\it Sensors}, Volume 14, Issue 3, Pages 5725-5741, 
Published by MDPI, ISSN 1424-8220, 
DOI 10.3390/s140305725, 
21 March 2014.

\item[Paper 2]
Niklas Karvonen, Denis Kleyko\\
{\bf "A Domain Knowledge-Based Solution for Human Activity Recognition: The UJA Dataset Analysis"}, 
In {\it Multidisciplinary Digital Publishing Institute Proceedings}, Vol. 2, Issue 19, Article number 1261, 
Published by MDPI, EISSN 2504-3900, 
DOI 10.3390/proceedings2191261, 
19 October 2018.

\item[Paper 3]
Niklas Karvonen, Lara Lorna Jimenez, Miguel Simon Gomez, Joakim Nilsson, Basel Kikhia, Josef Hallberg
{\\\bf "Classifier Optimized for Resource-constrained Pervasive Systems and Energy-efficiency"}, 
In {\it International Journal of Computational Intelligence Systems}, ISSN 1875-6891, E-ISSN 1875-6883, Vol. 10, no 1, p. 1272-1279, 
Published by Atlantis Press, 
DOI 10.2991/ijcis.10.1.86, 
2017.

\item[Paper 4]
Niklas Karvonen, Joakim Nilsson, Denis Kleyko, Lara Lorna Jimenez
{\\\bf "Low-Power Classification using FPGA - An Approach based on Cellular Automata and Hyperdimensional Computing"}, Submitted to {\it IEEE Pervasive Computing}, ISSN 1536-1268, 
November 2018

\item[Paper 5]
Anders Hedman, Niklas Karvonen, Josef Hallberg, Juho Merilahti
{\\\bf "Designing ICT for Health and Wellbeing - An Allostatic, Behavioral-Change Approach to a Monitoring and Coaching App"}, 
In {\it Proceedings of the 6th International Workshop on Ambient Assisted Living (IWAAL)}, Lecture Notes in Computer Science (Information Systems and Applications, incl. Internet/Web, and HCI), Ambient Assisted Living and Daily Activities, Volume 8868, Pages 244-251, 
Published by Springer, ISSN 0302-9743, Belfast, UK, 
2-5 December 2014.

\item[Paper 6]
Basel Kikhia, Thanos G. Stavropoulos, Stelios Andreadis, Niklas Karvonen, Ioannis Kompatsiaris, Stefan Sävenstedt, Marten Pijl, Catharina Melander\\
{\bf "Utilizing a Wristband Sensor to Measure the Stress Level for People with Dementia"}, 
In {\it Sensors}, ISSN 1424-8220, E-ISSN 1424-8220, Vol. 16, no 12, Article number 1989,
Published by MDPI, ISSN 1424-8220,
DOI https://doi.org/10.3390/s16121989,
24 November 2016

\item[Paper 7]
Kåre Synnes, Margareta Lilja, Anneli Nyman, Macarena Espinilla, Ian Cleland, Andres Gabriel Sanchez Comas, Zhoe Comas-Gonzalez, Josef Hallberg, Niklas Karvonen, Wagner Ourique de Morais, Federico Cruciani, Chris Nugent\\
{\bf "H2Al—The Human Health and Activity Laboratory"}, 
In {\it Multidisciplinary Digital Publishing Institute Proceedings}, ISSN 2504-3900, 2018, Vol. 2, no 19, Article number 1241,
Published by MDPI, ISSN 1424-8220,
DOI 10.3390/proceedings2191241,
14 November 2018
\end{description}

\subsection{Research Grants}
720.000 SEK funding from ALMI and Region Norrbotten for developing "Sensorizon", a commercialisation of the research results from the paper "Utilizing a Wristband Sensor to Measure the Stress Level for People with Dementia". I participated in the writing of the application and worked as the head of development within the project.

%\subsection{Research Collaborations}

%\subsection{Dissemination of Research Results}

\newpage

% Pedagogical merits
\section{Software Development}
I have a wide programming experience that includes working with microcontroller programming (including real-time systems), mobile applications, desktop applications, computer graphics, web applications, and algorithm development. Some of the languages I have used more commonly include: C, Java, Javascript, Python, and PHP. I also have experience working with both SQL and No-SQL databases. A more detailed description of various software projects I have been involved in can be provided on request.

% Pedagogical merits
\section{Pedagogics}
\subsection{Pedagogical Education}
University Pedagogics I - Luleå University of Technology (3.5 HP).\\
University Pedagogics II - Luleå University of Technology (4 HP).

\subsection{Teaching Experience}
Lab assistant - Object Oriented Programming. (LTU) \\
Lab assistant - Real-Time Systems. (LTU) \\
\newline
Lecturer - Dynamic Web Systems. (LTU) \\
Guest lecturer - Machine Learning and Web development (LTU)
Lecturer - ASP.NET programming (Strand Interconnect, CopyCat, Tanzania)
\newline
\newline
Supervisor - FAS 3 Kurt Kopppari, PHP programming (Trezer AB).\\
Master Thesis Supervisor - Tobias Axelsson, "Using supervised learning algorithms to model the behavior of Road Weather Information System sensors" (LTU). \\
\newline
Master Thesis Supervisor - Simon Nilsson Guldstrand, Maxime Koitsalu, "Re-imagining business processes in indoor locations with Smart Glasses and indoor positioning technology" (LTU). \\
Master Thesis Supervisor - Olof Enström, "Authentication Using Deep Learning on
User Generated Mouse Movement Images" (LTU)

\section{Management Experience}
Chief Executive Officer - Memorizon AB (2017-2018).\\
Chief Technical Officer - KYAB (2009-2010).

\section{Other Assignments}
\subsection{Board Representation}
Member of the board, Memorizon AB. \\
Advisory member of the board, Taxijakt AB\\
PhD Representative for SRT (LTU) in the PhD Association.

\section{References}
References from all my positions can be provided on request.

\end{document}
